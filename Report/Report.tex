\documentclass{article}
\usepackage{xeCJK, amsmath, amssymb, fontspec, geometry, color, enumerate, framed, CJKnumb}
\usepackage{float, subcaption}
\usepackage[cache=false]{minted}
\defaultCJKfontfeatures{AutoFakeBold=4, AutoFakeSlant=.4}
\setCJKmainfont{DFKai-SB.ttf}
\setmonofont{Courier.ttf}
\title{NASA 2017 Final Project \\ SA\#2 NFS Server Static Load Balance}
\author{B03901018\, 電機三\, 楊程皓\\ B03901078\, 電機三\, 蔡承佑\\ B05902053\, 資工一\, 陳奕均}
\everymath{\displaystyle}
\geometry{tmargin=20pt}
\begin{document}
\newcommand{\red}[1]{\textcolor{red}{#1}}
\newcommand{\br}[1]{\left( #1 \right)}
\newcommand{\sbr}[1]{\left[ #1 \right]}
\newfontfamily\cs{Courier.ttf}
\maketitle
\section{Introduction}
Currently, we have only one NFS server supplying all the workstation. Now suppose we have $N$ NFS server, 
providing $M$ users to access their home directory, we have to support following functions:
   \begin{enumerate}
   \item If a new grade of students joins, we have to use scripts to create their corresponding home directory on NFS.
   \item If new NFS servers joins, we have to adjust the home directories to balance the load.
   \item If there are old NFS servers discarded, we have to move the home directories to other NFS servers.
   \end{enumerate}
In this project, all VMs are under CentOS 7

\section{Workstation Environment Simulation}
\subsection{Structure}
\begin{figure}[H]
\centering
\includegraphics[height=0.2\textheight]{Fig1.png}
\caption{Structure of the workstation}
\end{figure}
\subsection{NFS Setup}
In NFS server, we must write the IP of the server in {\cs /etc/exports} to share directories to the workstation server.
It is important that the IP address of NFS server and workstation server must be constant.
In {\cs /etc/systemd/network/25-wired.network}, we add \\
\begin{minted}{shell}
[Match]
Name=enp0s8

[Network]
Address=192.168.100.3/24
\end{minted}
So that the IP addresses of NFS server and workstation server will be constant, which is the value given manually.
And the mount points are divided by grades of users. i.e. b03, b04, b05, etc. For example, in NFS-1:
\subsection{Server Setup}
First, to decrease system loading, we use {\cs autofs} to mount the NFS directories. Only when the mounted points are used will they be mounted. 
If a mount point is not used for a given time, it will be unmounted. 

\section{Creating One Single User}
There are several steps we must follow to create a user:
   \begin{enumerate}
   \item Give a user name and password
   \item Set the home directory
   \item Link the home directory to NFS
   \item Change the owner of the home directory of the user.
   \end{enumerate}
Part of our script to do this is shown as following:
\begin{framed}
\begin{minted}{shell}
useradd $username -d /home/$grade/$username
echo -e "$username\n$username\n" | passwd $username >& /dev/null
mv /home/$grade/$username /autofs/$nfs/$grade/$username
ln -s /autofs/$nfs/$grade/$username /home/$grade/$username
chown -R $username /autofs/$nfs/$grade/$username
\end{minted}
\end{framed}
After execution of this script, we create a new user whose home directory is a symbolic link to the directory lies in NFS server.

\section{Creating or Deleting A Group of Users}
Suppose now we want to create accounts for freshmen, e.g. b06902\{001-120\}. 
Or more generally, we want to create or remove tens of or hundres of users.
The only data now we have is the current user\_list and the current distribution.
All we need to do is shown in Figure \ref{userChange}.
\begin{figure}[h]
\centering
\includegraphics[width=0.7\textwidth]{Fig2.png}
\caption{flow diagram of creating/removing a group of users}
\label{userChange}.
\end{figure}
There are two main changes we have to manage: on {\bf NFS server} side (distribution) and {\bf workstation server} side
(add/remove home directories). To be more clearly, following are steps we should do:
   \begin{enumerate}
   \item Backup the old user\_list and generate a new user\_list base on the old one upon need.
   \item Call {\cs distribute.out} and input the modified user\_list and old distribution.
   This program will compare those files to redistribute the newly created user or remove users to balance the distribution.
   \item Call {\cs group\_user\_change} to compare the previous user\_list and modified user\_list to figure out
   the home directories links we have to modify.
   \item Call {\cs update\_users.sh} and input distribution/changes and group\_user\_change to complete the modifications on workserver server and NFS server.
   \end{enumerate}

\section{Creating New NFS Server}
Steps are as following:
   \begin{enumerate}
   \item Setup NFS server for the workstation server to connect to and mount.
   \item Generate a new distribution list and find the users that are to be moved.
   \item Move the directories of these users to the new server.
   \item Update the symbolic link of the home directories of these users.
   \end{enumerate}
\subsection{Setup NFS Server}
For administrator's convinence, we can {\cs ssh} to the NFS server from workstation server and setup directories to share in
{\cs /etc/exports}. Then setup the autofs at the workstation side ({\cs /etc/auto.master, /etc/auto.nfs*})
\subsection{Find The Users To Be Moved}
We write a C++ program to do this job. The input is the previous distribution list, and output would be a new distribution list and
the modification we should make. We evenly take some users from each server, and move them to the new server.
\subsection{Move The Directories}
\section{Deleting Old NFS Server}
Quite similar to adding new NFS server, but now we need not to setup a new server.
Steps are as following:
   \begin{enumerate}
   \item Decide the new distribution
   \item Move the directories to the new distributed NFS server
   \item Update the symbolic link of the home directories of these users.
   \end{enumerate}
\subsection{Decide The New Distribution}
Similar to adding new NFS server, we use a C++ program to distribute the users on the to-be-deleted server to other servers,
and figure out the users that are influenced.
\subsection{Move The Directories to New NFS Server}
\subsection{Update Symbolic Link}
\section{Difficulties}
Every single jobs look easy, but how to integrate them for more convinient use is the main problem of this program.
Though each steps looks easy, but we nead a clear structure to specify them. For example, I/O format should be 
accord between programs/scripts, and packeting all programs and scripts into a single script for administrator's convinence.
\end{document}
